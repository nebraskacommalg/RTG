\documentclass[12pt]{amsart}


\usepackage{times}
\usepackage[margin=0.8in]{geometry}
\usepackage{amsmath,amssymb,multicol,graphicx,framed,ifthen,color,xcolor,stmaryrd,enumitem,colonequals}



\definecolor{chianti}{rgb}{0.6,0,0}
\definecolor{meretale}{rgb}{0,0,.6}
\definecolor{leaf}{rgb}{0,.35,0}
\newcommand{\Q}{\mathbb{Q}}
\newcommand{\N}{\mathbb{N}}
\newcommand{\Z}{\mathbb{Z}}
\newcommand{\R}{\mathbb{R}}
\newcommand{\C}{\mathbb{C}}
\newcommand{\e}{\varepsilon}
\newcommand{\m}{\mathfrak{m}}
\newcommand{\inv}{^{-1}}
\newcommand{\ds}{\displaystyle}


\newcommand{\rsa}{\rightsquigarrow}


\newcommand\ceq{\colonequals}


\DeclareMathOperator{\ord}{ord}




%\begin{document}\begin{itemize}

%\thispagestyle{empty}




\begin{document}

	

	
\section*{Cheat sheet}

This is a \emph{listeners} guide to basic topics from Homological Algebra and Commutative Algebra \textsc{II} that appear in seminar. The only purpose of this is so that you might get a better chance of understanding the big picture of a talk that uses these concepts.

\subsection*{Equal / mixed characteristic} A local ring $(R,\m,k)$ has:
\begin{itemize}
\item \textit{equal characteristic zero} if $\mathrm{char}(R)=\mathrm{char}(k)=0$
\item \textit{equal characteristic $p$} if $\mathrm{char}(R)=\mathrm{char}(k)=p$ for some prime $p>0$
\item \textit{mixed characteristic $(0,p)$} if $\mathrm{char}(R)=0$ and $\mathrm{char}(k)=p$ for some prime $p>0$.
\end{itemize}
Any \emph{reduced} local ring satisfies one of the three conditions above; \emph{any} local ring satisfies one of the above or $\mathrm{char}(R)=p^n$ and $\mathrm{char}(k)=p$ for some prime $p$ and $n>1$.


\subsection*{Regular rings} When you hear \emph{regular}, you should think of localization of a polynomial ring or a power series ring. A Noetherian local ring $(R,\m,k)$ is \emph{regular} if $\dim(R) = \dim_{k-\text{vectorspace}}(\m/\m^2)$. Note that the inequality $\leq$ always holds by Krull height and NAK. A not-necessarily-local Noetherian ring is \emph{regular} if all of its localizations are. The basic examples are
\begin{itemize}
\item $K[X_1,\dots,X_n]$ is regular for a field $K$.
\item $K\llbracket X_1,\dots,X_n\rrbracket$ is regular for a field $K$.
\end{itemize}



\subsection*{Complete / completion} When you hear \emph{complete}, you should think of quotient of power series ring: the quintessential example of a complete local ring is $K\llbracket X_1,\dots,X_n \rrbracket / I$ for some field $K$ and ideal $I$.

In short, the \emph{completion} of a local ring $(R,\m,k)$ is the metric completion of $R$ equipped with the (pseudo)metric given by $\mathrm{dist}(r_1,r_2) = 2^{- \sup\{ n \ | \ r_1-r_2 \in \m^{n}\}}$; i.e., two elements are close if their difference is in a large power of $\m$. A ring is \emph{complete} if it is complete with respect to this (pseudo)metric. One can do a similar thing with $R$-modules.

Here are some key points:
\begin{itemize}
\item The completion of $K[X_1,\dots,X_n]_{(X_1,\dots,X_n)}/I$ is $K\llbracket X_1,\dots,X_n\rrbracket / I$ for a field $K$.
\item Every complete Noetherian local ring of equal characteristic is isomorphic to a ring of the form $K\llbracket X_1,\dots,X_n\rrbracket / I$ for some field $K$, and something similar is true in mixed characteristic.
\item A Noetherian local ring is ``closely connected'' to its completion (the map is faithfully flat, and with regular fibers in many cases).
\item Complete local rings have good technical properties (like stronger versions of NAK, Hensel's Lemma).
\end{itemize}



\subsection*{Exact sequences} An \emph{exact seqeunce} of $R$-modules is 
a (finite or infinite) collection of modules and maps
\[ \cdots  \rightarrow M_{i-1} \xrightarrow{d_{i-1}} M_{i} \xrightarrow{d_{i}} M_{i+1} \rightarrow \cdots\]
such that the kernel of any map $d_{i}$ is equal to the image of the previous map $d_{i-1}$. Special examples are \emph{short exact sequences}: exact sequences of the form
\[ 0  \rightarrow L \xrightarrow{\alpha} M \xrightarrow{\beta} N \rightarrow 0\]
the definition of exact in this case says that $\alpha$ is injective, $\beta$ is surjective, and $N\cong M/L$.

The main tricks to beware of in exact sequences are:
\begin{itemize}
\item if $\cdots \to  0 \to M \to 0 \to \cdots$ appears, then $M=0$.
\item if $\cdots \to  0 \to M \to N \to 0 \to \cdots$ appears, then $M\cong N$.
\item if $\cdots \to  0 \to M \xrightarrow{\alpha} N \to \cdots$ appears, then $\alpha$ is injective.
\item if $\cdots \to  M \xrightarrow{\beta} N \to 0 \to  \cdots$ appears, then $\beta$ is surjective.
\end{itemize}


\subsection*{Complexes and homology} A \emph{complex} of $R$-modules weakening of exact sequence: a complex is a (finite or infinite) collection of modules and maps
\[ \cdots  \rightarrow M_{i-1} \xrightarrow{d_{i-1}} M_{i} \xrightarrow{d_{i}} M_{i+1} \rightarrow \cdots\]
such that the composition of any two maps in a row is zero.

The $i$th \emph{homology} or \emph{cohomology} (you can think of these words as interchangable at first) of the complex above is the module $\ker(d_i)/\mathrm{im}(d_{i-1})$.

\subsection*{Free resolutions} A \emph{free resolution} of n $R$-module $M$ is a (finite or infinite) exact sequence of the form
\[ \cdots\to  R^{b_2} \rightarrow R^{b_1} \rightarrow R^{b_0}(  \rightarrow M) \to 0\]
for some $b_i$ (also possibly infinite).

\subsection*{Tensor products} The \emph{tensor product} of two $R$-modules $M$ and $N$ is another module $M\otimes_R N$ that satisfies a particular universal property. The key examples of tensor products are
\begin{itemize} 
\item $R/I \otimes_R R/J = R/(I+J)$
\item If $S$ is an $R$-algebra and $M$ is an $R$-module with presentation matrix $B$, then $S\otimes_R M$ is the $S$-module with presentation matrix $B$ (image of $B$ in $S$).
\end{itemize}


\subsection*{Ext and Tor} To any pair $M,N$ of $R$-modules, there is a sequence of $\mathrm{Ext}$ modules $\mathrm{Ext}^i_R(M,N)$. In short, they are defined by taking a free resolution of $M$, computing the module of homomorphisms into $N$ at each step to get a complex, and taking the homologies. One of the main tricks with $\mathrm{Ext}$ is that given a short exact sequence $0\to N' \to N \to N'' \to 0$, there is a long exact sequence
\[ \cdots \to \mathrm{Ext}^i_R(M,N') \to \mathrm{Ext}^i_R(M,N) \to \mathrm{Ext}^i_R(M,N'') \to \mathrm{Ext}^{i+1}_R(M,N') \to \cdots\]
Also, $\mathrm{Ext}^0_R(M,N)=\mathrm{Hom}_R(M,N)$, the module of $R$-linear homomorphisms from $M$ to $N$.

There is a sequence of $\mathrm{Tor}$ modules $\mathrm{Tor}_i^R(M,N)$. In short, they are defined by taking a free resolution of $M$, computing the module of homomorphisms into $N$ at each step to get a complex, and taking the homologies.
One of the main tricks with $\mathrm{Tor}$ is that given a short exact sequence ${0\to N' \to N \to N'' \to 0}$, there is a long exact sequence
\[ \cdots \to \mathrm{Tor}_i^R(M,N') \to \mathrm{Tor}_i^R(M,N) \to \mathrm{Tor}_i^R(M,N'') \to \mathrm{Tor}_i^R(M,N') \to \cdots\]
The other main trick with $\mathrm{Tor}$ is that $\mathrm{Tor}_i^R(M,N)\cong \mathrm{Tor}_i^R(N,M)$.



	











\end{document}
